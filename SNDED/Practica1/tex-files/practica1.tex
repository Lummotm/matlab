\documentclass{article}
\usepackage[utf8]{inputenc}
\usepackage[T1]{fontenc}
\usepackage{amsmath}
\usepackage{subcaption} % para poder poner una figura al lado de otra
\usepackage[margin=2.5cm]{geometry} % margenes a 2.5cm
\usepackage[spanish, provide=*]{babel}
\usepackage{graphicx}
\usepackage{xfrac} % más fracciones yipee
\usepackage{hyperref} % referencias
\usepackage{booktabs}

\title{Práctica 1}
\author{David Nikolov Yordanov}
\date{17 de Noviembre de 2025}

\begin{document}

\maketitle

\vspace{5cm}

\tableofcontents

\clearpage

\section{Introducción}
Este informe analiza la resolución numérica de la ecuación del calor mediante diferencias finitas.
Las tablas y algunas gráficas extra se han incluido dentro de los anexos.
\noindent \textbf{Notación:} \(J\) es el número de nodos espaciales con paso \(h = \sfrac{1}{J}\), y \(N\) el número de pasos temporales con paso \(k = \sfrac{T}{N}\). Definimos el operador de diferencias centrales espaciales como \(\delta_x^2 U_j = U_{j+1} - 2U_j + U_{j-1}\).

\section{Parte 1: Ecuación Homogénea}
Se resuelve \(\frac{\partial^2 u}{\partial x^2} = \frac{\partial u}{\partial t}\) con \(u(0,t)=u(1,t)=0\) y \(u(x,0) = \sin(2 \pi x)\).
Buscamos aproximaciones discretas \(\{U_{j}^n\}\) mediante los siguientes esquemas:

\noindent
\begin{minipage}[t]{0.48\textwidth}
	\textbf{1. Explícito:}
	\[ \frac{U_{j}^{n+1} - U_j^{n}}{k} = \frac{1}{h^2} \delta_x^2 U_j^n \]
\end{minipage}\hfill
\begin{minipage}[t]{0.48\textwidth}
	\textbf{2. Implícito:}
	\[ \frac{U_{j}^{n+1} - U_j^{n}}{k} = \frac{1}{h^2} \delta_x^2 U_j^{n+1} \]
\end{minipage}

\vspace{0.2cm}
\noindent \textbf{3. Crank-Nicolson:}
Definimos la aproximación en el paso intermedio como \( U_j^{n+1/2} = \frac{U_j^{n+1} + U_j^n}{2} \). El esquema es:
\[ \frac{U_{j}^{n+1} - U_j^{n}}{k} = \frac{1}{h^2} \delta_x^2 U_j^{n+1/2} \]

\subsection{Justificación Teórica del Error}
El error de truncamiento local \(T\) se obtiene sustituyendo la solución exacta \(u(x,t)\) en los esquemas numéricos.

\noindent \textbf{1. Error Espacial (Igual en los 3 métodos):}
Expandimos Taylor en espacio para \(u(x\pm h, t)\):
\[ u(x\pm h, t) = u(x,t) \pm h \frac{\partial u}{\partial x} + \frac{h^2}{2} \frac{\partial^2 u}{\partial x^2} \pm \frac{h^3}{6} \frac{\partial^3 u}{\partial x^3} + \frac{h^4}{24} \frac{\partial^4 u}{\partial x^4} + O(h^5) \]
Sumando ambas expresiones, los términos de derivada impar se cancelan:
\[ \frac{u(x+h, t) - 2u(x,t) + u(x-h, t)}{h^2} = \frac{\partial^2 u}{\partial x^2} + \frac{h^2}{12} \frac{\partial^4 u}{\partial x^4} + O(h^4) \]
Esto demuestra que la aproximación espacial es de orden \(O(h^2)\).

\vspace{0.2cm}
\noindent \textbf{2. Error Temporal - Explícito e Implícito (\(O(k)\)):}
Expandimos \(u(x, t+k)\) en torno a \(t\), obtenemos:
\[ \frac{u(x, t+k)-u(x, t)}{k} = \frac{\partial u}{\partial t} + \frac{k}{2}\frac{\partial^2 u}{\partial t^2} + O(k^2) \]
El término proporcional a \(k\) no se cancela, por lo que \(T\) es dominado por el error lineal.

\vspace{0.2cm}
\noindent \textbf{3. Error Temporal - Crank-Nicolson (\(O(k^2)\)):}
Expandimos Taylor alrededor del punto medio \(t_{n+1/2} = t_n + k/2\):
\begin{align*}
	u(x, t_{n+1}) & = u(x, t_{n+1/2}) + \frac{k}{2} \frac{\partial u}{\partial t}(x, t_{n+1/2}) + \frac{k^2}{8} \frac{\partial^2 u}{\partial t^2}(x, t_{n+1/2}) + O(k^3) \\
	u(x, t_{n})   & = u(x, t_{n+1/2}) - \frac{k}{2} \frac{\partial u}{\partial t}(x, t_{n+1/2}) + \frac{k^2}{8} \frac{\partial^2 u}{\partial t^2}(x, t_{n+1/2}) + O(k^3)
\end{align*}
Restando ambas ecuaciones, los términos pares se cancelan:
\[ \frac{u^{n+1} - u^n}{k} = \frac{\partial u}{\partial t}(x, t_{n+1/2}) + O(k^2) \]
Sumando ambas, aproximamos la función en el punto medio:
\[ \frac{u^{n+1} + u^n}{2} = u(x, t_{n+1/2}) + O(k^2) \]
Combinando esto con el error espacial, el error global \(T\) resulta \(O(k^2) + O(h^2)\).

\subsection{Análisis de Estabilidad de Von Neumann}
El análisis de estabilidad lineal estudia la propagación de errores a lo largo del tiempo. Definimos el error numérico \(\epsilon_j^n\) como la diferencia entre la solución numérica obtenida \(U_j^n\) y la solución exacta \(u(x_j, t_n)\):
\[
	\epsilon_j^n = U_j^n - u(x_j, t_n)
\]
Consideramos una perturbación en forma de modo de Fourier:
\[
	\epsilon_j^n = \xi^n e^{i \beta (jh)}
\]
donde \(\xi\) es el factor de amplificación y \(\beta\) el número de onda. Para que el método sea estable y los errores no crezcan descontroladamente, requerimos \( |\xi| \le 1 \).

\paragraph{1. Método Explícito:}
Sustituyendo en la ecuación en diferencias y definiendo \(\mu = k/h^2\), obtenemos:
\[ \xi = 1 - 4\mu \sin^2\left(\frac{\beta h}{2}\right) \]
Para garantizar \( |\xi| \le 1 \), es necesario que \( -1 \le 1 - 4\mu \), lo que implica la restricción de estabilidad condicional:
\[ \mu \le \frac{1}{2} \]

\paragraph{2. Método Implícito:}
La ecuación del error sigue el esquema \( (1 - \mu \delta_x^2) \epsilon^{n+1} = \epsilon^n \). En el dominio de frecuencia:
\[ \xi \left(1 + 4\mu \sin^2\left(\frac{\beta h}{2}\right)\right) = 1 \implies \xi = \frac{1}{1 + 4\mu \sin^2(\frac{\beta h}{2})} \]
Dado que el denominador es siempre \(\ge 1\) para todo \(\mu > 0\), se cumple siempre \( |\xi| \le 1 \). El método es \textbf{incondicionalmente estable}.

\paragraph{3. Método Crank-Nicolson:}
El error evoluciona según \( (1 - \frac{\mu}{2} \delta_x^2) \epsilon^{n+1} = (1 + \frac{\mu}{2} \delta_x^2) \epsilon^n \). El factor de amplificación es:
\[ \xi = \frac{1 - 2\mu \sin^2(\frac{\beta h}{2})}{1 + 2\mu \sin^2(\frac{\beta h}{2})} \]
Como \( |1 - A| \le |1 + A| \) para cualquier \(A \ge 0\), tenemos que \( |\xi| \le 1 \) para todo \(\mu > 0\). Es \textbf{incondicionalmente estable}.

\paragraph{Justificación del Término Fuente:}
En la Parte 2, la ecuación diferencial es \( u_t = u_{xx} + f(x,t) \). Tanto el esquema numérico discreto como la solución exacta contienen este término \( f(x,t) \) sumando en el lado derecho.
Al calcular la ecuación del error restando la expresión exacta de la numérica (\( \epsilon = U - u \)), el término \( f(x,t) \) aparece en ambos términos de la resta y se cancela exactamente (\( f - f = 0 \)).
En consecuencia, la evolución del error sigue estando gobernada exclusivamente por la parte homogénea de la ecuación, por lo que las condiciones de estabilidad demostradas anteriormente se aplican sin cambios.

\clearpage
\subsection{Análisis de Convergencia}
Para verificar los ordenes de convergencia teóricos se realizan dos tipos de tests.

\subsubsection{Convergencia Espacial (p)}
\noindent Para medir la convergencia espacial \(p\), se fija la relación \(\mu = \sfrac{k}{h^2} = 0.4\). Como se aprecia en la Figura~\ref{fig:conv_h_cn}, verificando el orden espacial de Crank-Nicolson. En los casos de implícito y explícito en estas condiciones se tiene que como \(k = 0.4\cdot h^2\) entonces \(E \approx O(0.4 \cdot h^2 + h^2) = O (h^2)\). Teniéndose un orden de convergencia espacial cuadrático aun siendo ambos de orden lineal en tiempo (véase la Figura~\ref{fig:conv_h_anexo}).

\begin{figure}[h!]
	\centering
	\includegraphics[width=0.60\textwidth]{assets/P1_convergencia_h_Metodo_3.png}
	\caption{P1 Convergencia espacial (p) para Crank-Nicolson. Pendiente \(p \approx 2.0\).}\label{fig:conv_h_cn}
\end{figure}


\noindent La Figura~\ref{fig:conv_h_no_ajustado} demuestra por qué el ajuste \(\mu=cte\) es necesario. Si no se ajusta (usando \(k \propto h\)), el error temporal \(O(k)\) del método implícito pasa a ser dominante sobre el error espacial \(O(h^2)\), y la pendiente global se aproxima a \(p \approx 1.43 \).
\begin{figure}[h!]
	\centering
	\includegraphics[width=0.60\textwidth]{assets/P1_convergencia_h_Metodo_2_no_ajustado.png}
	\caption{P1 Convergencia espacial (p) para Implícito sin ajuste \(\mu\). Pendiente \(p \approx 1.43\).}\label{fig:conv_h_no_ajustado}
\end{figure}


\clearpage

\subsubsection{Convergencia Temporal (q)}

\noindent Para medir el orden temporal \(q\), se aísla el error temporal usando una malla espacial muy fina (\(J=2000\)) para que el error \(O(h^2)\) sea despreciable. \\

\noindent \textbf{Nota sobre el método explícito:} Este test se omite para el método explícito. Su condición de estabilidad (\(\mu \le 1/2\))
fuerza a que \(k \propto h^2\). Esto ata el error temporal al espacial, impidiéndonos aislar el error temporal del espacial.


\noindent Las Figuras~\ref{fig:conv_k_cn} y \ref{fig:conv_k_imp} muestran los resultados de este test.
\begin{itemize}
	\item \textbf{Crank-Nicolson (Fig.~\ref{fig:conv_k_cn}):} Confirma claramente el resultado teórico esperado \(E \approx O(k^2)\), con una pendiente \(q \approx 2.0\).
	\item \textbf{Implícito (Fig.~\ref{fig:conv_k_imp}):} Muestra una pendiente \(q \approx 1.35\) bastante cercana a \(1\), refinando más la malla este valor tiende a \(1\) demostrando nuestro resultado teórico \(E \approx O(k)\).
\end{itemize}


\begin{figure}[h!]
	\centering
	\includegraphics[width=0.60\textwidth]{assets/P1_convergencia_k_Metodo_3.png}
	\caption{P1 Convergencia temporal (q) para Crank-Nicolson. Pendiente \(q \approx 2.0\).}\label{fig:conv_k_cn}
\end{figure}

\begin{figure}[h!]
	\centering
	\includegraphics[width=0.60\textwidth]{assets/P1_convergencia_k_Metodo_2.png}
	\caption{P1 Convergencia temporal (q) para Implícito. Pendiente \(q \approx 1.35\).}\label{fig:conv_k_imp}
\end{figure}

\clearpage

\subsection{Análisis de Eficiencia}


\noindent Finalmente, para comparar la eficiencia real de los tres métodos, se realiza una prueba híbrida. Para los métodos incondicionalmente estables (Implícito y Crank-Nicolson), se genera un conjunto de puntos probando una amplia cantidad de valores \(J \times N\). Para el método Explícito, se consideran valores estables \(\mu = 0.4\).

\noindent La Figura~\ref{fig:eficiencia_final} muestra los resultados en una gráfica log-log de Error vs. Tiempo de Cómputo.

\begin{figure}[ht]
	\centering
	\includegraphics[width=0.7\textwidth]{assets/P1_comparacion_eficiencia_GENERAL_RELATIVO.png}
	\caption{P1 Comparación de Eficiencia. Crank-Nicolson (verde), Implícito (azul) y Explícito (rojo).}\label{fig:eficiencia_final}
\end{figure}

\noindent De esta gráfica se extraen las conclusiones finales de la práctica:

\begin{itemize}
	\item \textbf{Crank-Nicolson (Verde):} El conjunto de puntos verdes forma la ``frontera eficiente''. Sus puntos están sistemáticamente más abajo (menor error) y más a la izquierda (menor tiempo) que los demás. Es, sin duda, el método más eficiente.

	\item \textbf{Implícito (Azul):} Los puntos azules se sitúan por encima de los verdes. Demuestra ser una alternativa estable, pero para alcanzar un nivel de error similar al de Crank-Nicolson, requiere un tiempo de cómputo notablemente mayor. Esto se debe a su convergencia temporal de orden \(q=1\).


	\item \textbf{Explícito (Rojo):} Los puntos rojos confirman que, aunque es preciso (error bajo), está muy desplazado a la derecha (tiempo de cómputo alto). Esto se debe a la condición de estabilidad (\(N \propto J^2\)), que obliga a un número excesivo de pasos de tiempo y dispara el coste, volviéndolo ineficiente. Se apreciará posteriormente con el caso del problema no homogéneo \hyperlink{conclusion_p2}{\textbf{[1]}}, la ineficacia del método.

\end{itemize}

\noindent \textbf{Conclusión:} El método de Crank-Nicolson (\(O(k^2+h^2)\)) es superior en todos los aspectos, ofreciendo la mayor precisión en el menor tiempo. El método Explícito (\(O(k+h^2)\)) es demasiado lento debido a su restricción de estabilidad, y el Implícito (\(O(k+h^2)\)), aunque estable, es menos eficiente que Crank-Nicolson por su convergencia temporal de primer orden. \\ \\

\noindent \textbf{Nota:} Las gráficas de eficiencia solicitadas en el enunciado (para \(k\) fijo variando \(h\) y viceversa) se han incluido en el Anexo (véanse las Figuras~\ref{fig:anexo_k_fijo} y~\ref{fig:anexo_h_fijo}). Se aprecia en ellas lo siguiente: en las gráficas \textbf{h-fijo} (Fig.~\ref{fig:anexo_h_fijo}), el error de \textbf{Crank-Nicolson} es casi plano (dominado por el ``suelo'' espacial \(O(h^2)\)) mientras el \textbf{Implícito} muestra la caída de su error temporal \(O(k)\). En las gráficas \textbf{k-fijo} (Fig.~\ref{fig:anexo_k_fijo}), se ve el error espacial \(O(h^2)\) disminuir hasta chocar con el ``suelo'' temporal \(O(k^q)\). Finalmente, el \textbf{Explícito} (`h-fijo`) muestra el ``rebote'' del error de redondeo debido al número excesivo de pasos (\(N \propto J^2\)).

\clearpage
\section{Ecuación No Homogénea}

\noindent Se considera ahora el problema no homogéneo:
\[
	u_t = u_{xx} + f(x,t)
	\quad \text{con} \quad
	0 < x < 1, \quad t > 0
\]
\noindent Con condiciones de borde y de inicio dadas por:
\begin{align}
	u(x,0) & = 0, \quad 0 \le x \le 1 \\
	u(0,t) & = t / (t+1), \quad t > 0 \\
	u(1,t) & = 0, \quad t > 0
\end{align}
\noindent Y un término fuente \(f(x,t)\) tal que la solución exacta es \(u(x,t) = \frac{t}{t+1}\cos(\pi x/2)^2\).

\noindent Se repite el análisis de la Parte 1 para este problema. Como se observará, la adición de un término fuente y condiciones de borde no homogéneas no altera las propiedades de convergencia ni la eficiencia relativa de los métodos.

\noindent \textbf{Nota:} Las gráficas de eficiencia (`k-fijo` y `h-fijo`) para esta parte también se incluyen en el Anexo (véanse las Figuras~\ref{fig:anexo_k_fijo_p2} y~\ref{fig:anexo_h_fijo_p2}).

\subsection{Justificación Teórica del Error (Caso No Homogéneo)}

Para la ecuación no homogénea \( u_t = u_{xx} + f(x,t) \), analizamos el error de truncamiento local \( T \) sustituyendo la solución exacta \( u \) en el esquema numérico y restando el término fuente discreto.

\subsubsection*{Métodos Explícito e Implícito}
Sustituyendo las expansiones de Taylor alrededor del punto \((x_j, t_n)\) en la definición del error de truncamiento:
\begin{align*}
	T_j^n & = \left[ \frac{u^{n+1} - u^n}{k} - \frac{1}{h^2}\delta_x^2 u \right] - f(x_j, t_n)                                        \\
	      & = \left[ u_t + \frac{k}{2}u_{tt} + O(k^2) \right] - \left[ u_{xx} + \frac{h^2}{12}u_{xxxx} + O(h^4) \right] - f(x_j, t_n) \\
	      & = \left( u_t - u_{xx} - f \right) + \frac{k}{2}u_{tt} - \frac{h^2}{12}u_{xxxx} + \dots
\end{align*}
Como \(u(x,t)\) es la solución exacta del problema, satisface la ecuación diferencial \( u_t - u_{xx} - f = 0 \). Por tanto, el primer paréntesis desaparece:
\[
	T_j^n = 0 + O(k) + O(h^2)
\]

\subsubsection*{Método de Crank-Nicolson}
El esquema promedia el término fuente en el tiempo. Utilizando las expansiones de Taylor centradas en \((x_j, t_{n+1/2})\):
\begin{align*}
	T_j^{n+1/2} & = \left[ \frac{u^{n+1} - u^n}{k} - \frac{1}{2h^2}\delta_x^2(u^{n+1} + u^n) \right] - \frac{f^{n+1} + f^n}{2}                                   \\
	            & = \left[ u_t + O(k^2) \right] - \left[ u_{xx} + O(k^2) + O(h^2) \right] - \underbrace{\left[ f + O(k^2) \right]}_{\text{ajustar al Taylor } f} \\
	            & = \left( u_t - u_{xx} - f \right) + O(k^2) + O(h^2)
\end{align*}
Nuevamente, el término agrupado \((u_t - u_{xx} - f)\) es exactamente cero por la definición de la ecuación diferencial. El error resultante conserva el orden cuadrático:
\[
	T_j^{n+1/2} = O(k^2) + O(h^2)
\]

\clearpage

\subsection{Análisis de Convergencia (Parte 2)}

\subsubsection{Convergencia Espacial (p)}
\noindent Se repite el test fijando \(\mu = 0.4\). Como se aprecia en la Figura~\ref{fig:conv_h_cn_p2}, Crank-Nicolson mantiene el orden \(p \approx 2.0\). Los métodos implícito y explícito (Figura~\ref{fig:conv_h_anexo_p2}) también lo hacen, ya que \(E \approx O(h^2)\).

\begin{figure}[h!]
	\centering
	\includegraphics[width=0.60\textwidth]{assets/P2_convergencia_h_Metodo_3.png}
	\caption{P2 Convergencia espacial (p) para Crank-Nicolson. Pendiente \(p \approx 2.0\).}\label{fig:conv_h_cn_p2}
\end{figure}

\noindent De nuevo, la Figura~\ref{fig:conv_h_no_ajustado_p2} demuestra que sin el ajuste \(\mu=cte\), el error temporal \(O(k)\) del método implícito domina, y la pendiente se aproxima a \(p \approx 0.91\).

\begin{figure}[h!]
	\centering
	\includegraphics[width=0.60\textwidth]{assets/P2_convergencia_h_Metodo_2_no_ajustado.png}
	\caption{P2 Convergencia espacial (p) para Implícito sin ajuste \(\mu\). Pendiente \(p \approx 0.91\).}\label{fig:conv_h_no_ajustado_p2}
\end{figure}

\clearpage
\subsubsection{Convergencia Temporal (q)}

\noindent Se repite el test con \(J=2000\) para aislar el error temporal.
\noindent Las Figuras~\ref{fig:conv_k_cn_p2} y \ref{fig:conv_k_imp_p2} confirman los resultados teóricos: \(E \approx O(k^2)\) para el método Crank-Nicolson (con \(q \approx 1.95\)) y \(E \approx O(k)\) para el método Implícito (con \(q \approx 0.93\)).

\begin{figure}[h!]
	\centering
	\includegraphics[width=0.60\textwidth]{assets/P2_convergencia_k_Metodo_3.png}
	\caption{P2 Convergencia temporal (q) para Crank-Nicolson. Pendiente \(q \approx 1.95\).}\label{fig:conv_k_cn_p2}
\end{figure}

\begin{figure}[h!]
	\centering
	\includegraphics[width=0.60\textwidth]{assets/P2_convergencia_k_Metodo_2.png}
	\caption{P2 Convergencia temporal (q) para Implícito. Pendiente \(q \approx 0.93\).}\label{fig:conv_k_imp_p2}
\end{figure}

\clearpage
\subsection{Análisis de Eficiencia (Parte 2)}

\noindent Finalmente, se repite la prueba de eficiencia híbrida. La Figura~\ref{fig:eficiencia_final_p2} muestra unos resultados análogos a la Parte 1.

\begin{figure}[ht]
	\centering
	\includegraphics[width=0.7\textwidth]{assets/P2_comparacion_eficiencia_GENERAL_RELATIVO.png}
	\caption{P2 Comparación de Eficiencia. Crank-Nicolson (verde), Implícito (azul) y Explícito (rojo).}\label{fig:eficiencia_final_p2}
\end{figure}

\noindent Las conclusiones son idénticas:
\begin{itemize}
	\item \textbf{Crank-Nicolson (Verde):} Sigue formando la ``frontera eficiente'', siendo el método más rápido para cualquier error deseado.
	\item \textbf{Implícito (Azul):} Sus puntos se sitúan por encima de los verdes, confirmando que es estable pero menos eficiente.
	\item \textbf{Explícito (Rojo):} Sus puntos son precisos (error bajo) pero lentos (tiempo de cómputo alto), confirmando que es inviable por su restricción de estabilidad.
\end{itemize}

\hypertarget{conclusion_p2}{}
\noindent \textbf{Conclusión (Parte 2):} El método de Crank-Nicolson \( O(k^2+h^2)\) sigue siendo superior. La presencia de un término fuente y condiciones de borde no homogéneas no altera la eficiencia relativa ni el orden de convergencia.

\noindent De hecho, este caso hace más visible la superioridad de Crank-Nicolson. Al no tender la solución a cero (como en el caso homogéneo), la prueba es más exigente. Esto resalta la ineficiencia del método explícito (cuyo coste computacional en el régimen estable es prohibitivo) y consolida a Crank-Nicolson como la opción más robusta y eficiente.


\clearpage
\section{Anexo - 1}

\begin{figure}[h!]
	\centering
	\begin{subfigure}[b]{0.48\textwidth}
		\centering
		\includegraphics[width=\textwidth]{assets/P1_convergencia_h_Metodo_1.png}
		\caption{P1 Convergencia espacial (p) - Explícito}\label{fig:conv_h_exp}
	\end{subfigure}
	\hfill
	\begin{subfigure}[b]{0.48\textwidth}
		\centering
		\includegraphics[width=\textwidth]{assets/P1_convergencia_h_Metodo_2.png}
		\caption{P1 Convergencia espacial (p) - Implícito}\label{fig:conv_h_imp}
	\end{subfigure}

	\caption{P1 Convergencia espacial para los métodos Explícito e Implícito con \(\mu=0.4\).}\label{fig:conv_h_anexo}
\end{figure}

\begin{figure}[h!]
	\centering
	\begin{subfigure}[b]{0.32\textwidth}
		\includegraphics[width=\textwidth]{assets/P1_k_fijo_var_h_Explícito.png}
		\caption{k-fijo, Explícito}
	\end{subfigure}
	\hfill
	\begin{subfigure}[b]{0.32\textwidth}
		\includegraphics[width=\textwidth]{assets/P1_k_fijo_var_h_Implícito.png}
		\caption{k-fijo, Implícito}
	\end{subfigure}
	\hfill
	\begin{subfigure}[b]{0.32\textwidth}
		\includegraphics[width=\textwidth]{assets/P1_k_fijo_var_h_Crank-Nicolson.png}
		\caption{k-fijo, Crank-Nicolson}
	\end{subfigure}

	\caption{P1 Gráficas de eficiencia con k-fijo.}\label{fig:anexo_k_fijo}
\end{figure}

\begin{figure}[h!]
	\centering
	\begin{subfigure}[b]{0.32\textwidth}
		\includegraphics[width=\textwidth]{assets/P1_h_fijo_var_k_Explícito.png}
		\caption{h-fijo, Explícito}
	\end{subfigure}
	\hfill
	\begin{subfigure}[b]{0.32\textwidth}
		\includegraphics[width=\textwidth]{assets/P1_h_fijo_var_k_Implícito.png}
		\caption{h-fijo, Implícito}
	\end{subfigure}
	\hfill
	\begin{subfigure}[b]{0.32\textwidth}
		\includegraphics[width=\textwidth]{assets/P1_h_fijo_var_k_Crank-Nicolson.png}
		\caption{h-fijo, Crank-Nicolson}
	\end{subfigure}

	\caption{P1 Gráficas de eficiencia con h-fijo.}\label{fig:anexo_h_fijo}
\end{figure}

\clearpage
\clearpage
\section{Anexo - 2}

\begin{figure}[h!]
	\centering
	\begin{subfigure}[b]{0.48\textwidth}
		\centering
		\includegraphics[width=\textwidth]{assets/P2_convergencia_h_Metodo_1.png}
		\caption{P2 Convergencia espacial (p) - Explícito}\label{fig:conv_h_exp_p2}
	\end{subfigure}
	\hfill
	\begin{subfigure}[b]{0.48\textwidth}
		\centering
		\includegraphics[width=\textwidth]{assets/P2_convergencia_h_Metodo_2.png}
		\caption{P2 Convergencia espacial (p) - Implícito}\label{fig:conv_h_imp_p2}
	\end{subfigure}
	\caption{P2 Convergencia espacial para Explícito e Implícito con \(\mu=0.4\).}\label{fig:conv_h_anexo_p2}
\end{figure}

\begin{figure}[h!]
	\centering
	\begin{subfigure}[b]{0.32\textwidth}
		\includegraphics[width=\textwidth]{assets/P2_k_fijo_var_h_Explícito.png}
		\caption{k-fijo, Explícito (P2)}
	\end{subfigure}
	\hfill
	\begin{subfigure}[b]{0.32\textwidth}
		\includegraphics[width=\textwidth]{assets/P2_k_fijo_var_h_Implícito.png}
		\caption{k-fijo, Implícito (P2)}
	\end{subfigure}
	\hfill
	\begin{subfigure}[b]{0.32\textwidth}
		\includegraphics[width=\textwidth]{assets/P2_k_fijo_var_h_Crank-Nicolson.png}
		\caption{k-fijo, Crank-Nicolson (P2)}
	\end{subfigure}

	\caption{P2 Gráficas de eficiencia con k-fijo.}\label{fig:anexo_k_fijo_p2}
\end{figure}

\begin{figure}[h!]
	\centering
	\begin{subfigure}[b]{0.32\textwidth}
		\includegraphics[width=\textwidth]{assets/P2_h_fijo_var_k_Explícito.png}
		\caption{h-fijo, Explícito (P2)}
	\end{subfigure}
	\hfill
	\begin{subfigure}[b]{0.32\textwidth}
		\includegraphics[width=\textwidth]{assets/P2_h_fijo_var_k_Implícito.png}
		\caption{h-fijo, Implícito (P2)}
	\end{subfigure}
	\hfill
	\begin{subfigure}[b]{0.32\textwidth}
		\includegraphics[width=\textwidth]{assets/P2_h_fijo_var_k_Crank-Nicolson.png}
		\caption{h-fijo, Crank-Nicolson (P2)}
	\end{subfigure}

	\caption{P2 Gráficas de eficiencia con h-fijo.}\label{fig:anexo_h_fijo_p2}
\end{figure}

\section{Anexo 3: Analisis de convergencia (Tablas)}

A continuación se presentan los resultados detallados del análisis de convergencia, calculando el Orden Experimental de Convergencia local mediante la fórmula:
\[
	p \approx \frac{\log(E_{i-1}/E_i)}{\log(h_{i-1}/h_i)}
\]

\subsection{Convergencia Espacial (Problema 1)}
Método: Crank-Nicolson. Se observa una convergencia cuadrática perfecta ($p \to 2$) a medida que se refina la malla.

\begin{table}[h!]
	\centering
	\begin{tabular}{|ccccc|}
		\toprule
		$J$ & $h$     & Error    & Ratio & Orden ($p$) \\
		\midrule
		50  & 0.02000 & 2.62e-02 & -     & -           \\
		100 & 0.01000 & 6.51e-03 & 4.03  & 2.0107      \\
		200 & 0.00500 & 1.62e-03 & 4.01  & 2.0027      \\
		400 & 0.00250 & 4.06e-04 & 4.00  & 2.0007      \\
		800 & 0.00125 & 1.01e-04 & 4.00  & 2.0002      \\
		\bottomrule
	\end{tabular}
	\caption{Análisis de convergencia espacial para $u_t = u_{xx}$.}
	\label{tab:p1_espacial}
\end{table}

\subsection{Convergencia Temporal (Problema 1)}
Comparativa de error absoluto. Se confirma que el método Implícito es de primer orden ($q \approx 1$) y Crank-Nicolson de segundo orden ($q \approx 2$).

\begin{table}[h!]
	\centering
	\begin{tabular}{|cc|cc|cc|}
		\toprule
		     &         & \multicolumn{2}{c|}{\textbf{Implícito}} & \multicolumn{2}{c}{\textbf{Crank-Nicolson}}                             \\
		$N$  & $k$     & Error (Abs)                             & Orden ($q$)                                 & Error (Abs) & Orden ($q$) \\
		\midrule
		100  & 0.00500 & 3.36e-02                                & -                                           & 1.20e-03    & -           \\
		200  & 0.00250 & 1.74e-02                                & 0.9449                                      & 2.99e-04    & 2.0057      \\
		400  & 0.00125 & 8.89e-03                                & 0.9713                                      & 7.44e-05    & 2.0055      \\
		800  & 0.00063 & 4.49e-03                                & 0.9851                                      & 1.84e-05    & 2.0180      \\
		1600 & 0.00031 & 2.26e-03                                & 0.9926                                      & 4.36e-06    & 2.0731      \\
		\bottomrule
	\end{tabular}
	\caption{Análisis de convergencia temporal para $u_t = u_{xx}$.}
	\label{tab:p1_temporal}
\end{table}

\subsection{Convergencia Temporal (Problema 2)}
Análisis con término fuente $f(x,t)$. Se mantiene la consistencia de los órdenes de convergencia.

\begin{table}[h!]
	\centering
	\begin{tabular}{|cc|cc|cc|}
		\toprule
		     &         & \multicolumn{2}{c|}{\textbf{Implícito}} & \multicolumn{2}{c}{\textbf{Crank-Nicolson}}                             \\
		$N$  & $k$     & Error (Rel)                             & Orden ($q$)                                 & Error (Rel) & Orden ($q$) \\
		\midrule
		100  & 0.00500 & 4.12e-03                                & -                                           & 1.10e-05    & -           \\
		200  & 0.00250 & 2.21e-03                                & 0.8962                                      & 2.89e-06    & 1.9311      \\
		400  & 0.00125 & 1.16e-03                                & 0.9332                                      & 7.46e-07    & 1.9562      \\
		800  & 0.00063 & 5.97e-04                                & 0.9577                                      & 1.90e-07    & 1.9720      \\
		1600 & 0.00031 & 3.04e-04                                & 0.9736                                      & 4.82e-08    & 1.9812      \\
		\bottomrule
	\end{tabular}
	\caption{Análisis de convergencia temporal para $u_t = u_{xx} + f(x,t)$.}
	\label{tab:p2_temporal}
\end{table}

\end{document}
